\documentclass[11pt]{article} % use larger type; default would be 10pt

\usepackage[utf8]{inputenc} % set input encoding (not needed with XeLaTeX)


%%% PAGE DIMENSIONS
\usepackage{geometry} % to change the page dimensions
\geometry{a4paper} % or letterpaper (US) or a5paper or....


\usepackage{graphicx} % support the \includegraphics command and options

% \usepackage[parfill]{parskip} % Activate to begin paragraphs with an empty line rather than an indent

%%% PACKAGES
\usepackage{booktabs} % for much better looking tables
\usepackage{array} % for better arrays (eg matrices) in maths
\usepackage{paralist} % very flexible & customisable lists (eg. enumerate/itemize, etc.)
\usepackage{verbatim} % adds environment for commenting out blocks of text & for better verbatim
\usepackage{subfig} % make it possible to include more than one captioned figure/table in a single float
\usepackage{cite}
% These packages are all incorporated in the memoir class to one degree or another...

%%% HEADERS & FOOTERS
\usepackage{fancyhdr} % This should be set AFTER setting up the page geometry
\pagestyle{fancy} % options: empty , plain , fancy
\renewcommand{\headrulewidth}{0pt} % customise the layout...
\lhead{}\chead{}\rhead{}
\lfoot{}\cfoot{\thepage}\rfoot{}

%%% SECTION TITLE APPEARANCE
\usepackage{sectsty}
\allsectionsfont{\sffamily\mdseries\upshape} % (See the fntguide.pdf for font help)
% (This matches ConTeXt defaults)

%%% ToC (table of contents) APPEARANCE
\usepackage[nottoc,notlof,notlot]{tocbibind} % Put the bibliography in the ToC
\usepackage[titles,subfigure]{tocloft} % Alter the style of the Table of Contents
\renewcommand{\cftsecfont}{\rmfamily\mdseries\upshape}
\renewcommand{\cftsecpagefont}{\rmfamily\mdseries\upshape} % No bold!


\title{Mapping the Brain: An Introduction to Connectomics\\Progress Report: Segmenting Mitochondria}
\author{Brandon Duderstadt, Ryan Marren, Eric Huang}
%\date{} % Activate to display a given date or no date (if empty),
         % otherwise the current date is printed 

\begin{document}
\maketitle

\section{Summary}

We spent the first week reading papers in order to get a better idea of what others did to segment mitochondria. Ultimately, we decided to emulate the methods used in "Supervoxel-Based Segmentation of Mitochondria..." (A. Lucchi, et. al). In the second week, Ryan built a pipeline that retrieves data, applies the SLIC supervoxel segmentation algorithm outlined in the Lucchi paper, and runs a random forest classifier and outputs segmented mitochondria and their graphs. The program is not perfect as it only segmented three mitochondria out of the hundreds in the image. We are currently varying the features in the machine learning algorithm in order to determine which detects mitochondria more accurately. Everything was done so far using Python.

\section{Updated Goals}
Our future goals are to stich the images together to create 3D mitochondria and perhaps test different pre-processing algorithms besides SLIC. Hopefully by the end of next week, we will have segmented mitochondria using a couple methods and are able to compare the results. We also hope to improve our algorithm to not cut out mitochondria excessively in the pre-processing stage as they are impossible to retrieve afterward.

\section{Updated Timeline}
End of Week 2: Improve segmentation algorithm. Work on pre-processing. Experiment a little with other methods
\newline
\newline
Week 3: Precision recall to compare methods.

\bibliography{yourbibname}{}
\bibliographystyle{plain}

\end{document}

























